\chapter{Conclusão}
\label{c.conclusao}

O objetivo deste projeto foi desenvolver um aplicativo modular que possa ser usado como ferramenta para o desenvolvimento de tecnologias web disponíveis atualmente, assim como o de propagar o conhecimento e facilitar a compreensão do assunto que envolve o tema de reconhecimento de sons. 

O código de todo o projeto desenvolvido foi disponibilizado no \emph{Github}, sob licença MIT, que garante permissão de cópia, reprodução, distribuição, publicação e venda para qualquer pessoa, reforçando o caráter público e comunitário. Para acesso ao código fonte do projeto e documentos utilizados durante o desenvolvimento acesse \url{https://github.com/evandrododo/piratify}.

Todas as etapas foram essenciais para a aplicação, e por se tratar de uma tecnologia em constante desenvolvimento a comunidade de programadores envolvidos contribuiu com as bibliotecas utilizadas tanto antes do início do projeto como durante, aplicando correções necessárias e modificações para melhoria de desempenho. O Resultado final foi uma aplicação de um modelo teórico simplificado e um protótipo funcional de reconhecimento de audio.